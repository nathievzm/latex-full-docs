\documentclass{article}

\usepackage{amsmath} % Load amsmath to access the \cfrac{...}{...} command
\usepackage[textwidth=9.5cm]{geometry}

\begin{document}

Fractions can be nested but, in this example, note how the default math styles, as used in the denominator, don't produce ideal results...

\[ \frac{1+\frac{a}{b}}{1+\frac{1}{1+\frac{1}{a}}} \]

\noindent ...so we use \verb|\displaystyle| to improve typesetting:

\[ \frac{1+\frac{a}{b}} {\displaystyle 1+\frac{1}{1+\frac{1}{a}}} \]

Here is an example which uses the \texttt{amsmath} \verb|\cfrac| command:

\[
    a_0+\cfrac{1}{a_1+\cfrac{1}{a_2+\cfrac{1}{a_3+\cdots}}}
\]

Here is another example, derived from the \texttt{amsmath} documentation, which demonstrates left
and right placement of the numerator using \verb|\cfrac[l]| and \verb|\cfrac[r]| respectively:
\[
    \cfrac[l]{1}{\sqrt{2}+
        \cfrac[r]{1}{\sqrt{2}+
            \cfrac{1}{\sqrt{2}+\dotsb}}}
\]

\vspace{16px}

\newcommand*{\contfrac}[2]{%
    {
            \rlap{$\dfrac{1}{\phantom{#1}}$}%
            \genfrac{}{}{0pt}{0}{}{#1+#2}%
        }
}
\[
    a_0 +
    \contfrac{a_1}{
        \contfrac{a_2}{
            \contfrac{a_3}{
                \genfrac{}{}{0pt}{0}{}{\ddots}
            }}}
\]

\end{document}